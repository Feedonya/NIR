\documentclass[12pt]{article}
\usepackage[russian]{babel}
\usepackage{color}.
\usepackage{ulem}\normalem.
\begin{document}
\bfseries
\begin{flushleft} 
\section{EquationEditor}
 – это средство визуального редактирования, предоставляющее набор \textcolor{red}{стандартных математических конструкций, которые вы можете заполнять числами},\textbf{ специальными} {\rm символами} символами символами и другими структурными частями формул.Чтобы отредактировать одну из формул, созданных{\bf с помощью EquationEditor,} \end{flushleft} \underline{достаточно дважды щелкнуть по ней или, выделив формулу}, \uppercase {выбрать команду Правка ? Объект Equation, а из появившегося подменю выбрать} \fbox{пункт Изменить. Это приведет} к запуску EquationEditor и вставке в {\Huge него} выбранной формулы для редактирования.\begin{flushright} Примеры математических{\tiny формул} , приведены в этой главе, изображены так, как вы видите их на экране. EquationEditor предоставляет немало \textit{мощных средств для настройки внешнего} вида \sout {и процесса набора} \xout {формул. В то же} время \uwave {стандартные настройки и стили EquationEditor подходят для большинства} математических, научных и деловых  работ.\end{flushright} Создание формулы напоминает сборку трехмерной
 \begin{itemize}
\item  головоломки: соединяя составные части по одной
\item вы стремитесь создать завершенную форму, к примеру, шар или
\item куб.
\end{itemize}
   Если одна из составных частей установлена неверно, то конечного результата вам достигнуть не удастся.
\begin{description}
\item[lorem:]
пункты помечаются маркерами;
\item[ipsum:]
пункты нумеруются;
\item[description:]
пункты снабжаются заголовками.
\end{description}

\end{document}